\chapter{Groupes libres et pr\'{e}sentations}
	Notion clef du M1 d'alg\`{e}bre :\textit{le quotient}, le passage au quotient , les applications \textit{bien d\'{e}finies}.
	\fbox{Contre-exemple}: On considère 
	\[ u : \Set{R}^2 \rightarrow \Set{R}^3 , xe + yf + zg \mapsto (x,y,z) \]
	Avec : \[ e = (1,0) \  f = (0,1) \  g = (1,1) \  x,y,z \in \Set{R} \]
	Ce n'est pas \textit{bien défini}, en effet on peut assoscier à $(0,0)$ le vecteur $(0,0,0)$ mais aussi le vecteur $(-1,-1,1)$.
	Dans un e.v de dimension finie, on sait qu'il existe des bases.
	\begin{theorem}
		Soit $\mathscr{B} = (e_i)_{1 \le i \le n}$ une base d'un e.v $E$. Soit $F$ un e.v et $f_1,\cdots,f_n$ des vecteurs de $F$. \newline
		Alors $\exists ! \varphi \in \mathscr{L}(E,F)$ tel que $\varphi(e_i) = f_i , \forall i \in \segN{1}{n}$
	\end{theorem}
	\begin{remark}
		Dans l'exemple pr\'{e}c\'{e}dant, l'application $\varphi \in \mathscr{L}(\Set{R}^2,\Set{R}^3)$ tel que 
		\[ \varphi(e) = (1,0,0) \ , \varphi(f) = (0,1,0) \ , \varphi(g) = (0,0,1) \]
		n'existe pas car $(e,f,g)$ n'est pas une base. Pour les groupes, il n'y a en g\'{e}n\'{e}ral pas de bases. \newline
		\fbox{But de la le\c{c}on} Remplacer le th\'{e}or\`{e}me sur les bases par un th\'{e}or\`{e}me adapt\'{e}. Introduire les groupes
		libres (un groupe o\`{u} il existe une famille libre et g\'{e}n\'{e}ratrice).
    \end{remark} 
	\section{Introduction aux groupes libres. Le groupe $\Set{Z}$}
	$(\Set{Z} , +)$ est un groupe
	\begin{proposition}
		Soit $G$ un groupe quelconque et $g$ dans $G$. Alors il existe un unique morphisme de groupes de $\Set{Z}$ dans $G$ qui envoie $1$ sur $g$
	\end{proposition}
	Id\'{e}e : $1$ joue le r\^{o}le d'une base.
	\begin{proof}
		Soit $\varphi : (\Set{Z}, +) \rightarrow (G, \cdot) , 1 \mapsto g$ \newline
		\fbox{Analyse}	$1+1 = 2 \mapsto g \cdot g = g^2$ \, car $\varphi(1+1) = \varphi(1) \cdot \varphi(1)$. \newline
		De m\^{e}me $n \mapsto g^n , \forall n > 0$ et $0 \mapsto e $ car l'image d'un neutre pour un morphisme est un neutre. \newline
		$-1 \mapsto g^{-1}$ car l'image de l'inverse est l'inverse de l'image , $-n \mapsto g^{-n}, \forall n > 0$.	\newline
		\fbox{Synth\`{e}se} On pose $\varphi(n) = g^n, \forall n \in \Set{Z}$ , on a bien $\varphi(1) = g$ , $\varphi(n+m) = g^{n+m} = g^n \cdot g^m 
		= \varphi(n) \cdot \varphi(m)$. \newline
		Donc $\varphi$ existe et son unicit\'{e} provient de la partie "analyse".
	\end{proof}
	\begin{definition}
		Le groupe $G$ est dit monog\`ene s'il est engendr\'e par un seul g\'en\'erateur $g$. \newline i.e $G = \{ g^n , n \in \Set{Z} \}$
	\end{definition}
	\begin{remark}
		Dans ce cas, on peut trouver un unique morphisme $\varphi$ tel que $\varphi(1) = g, \varphi : \Set{Z} \rightarrow G$ , ce morphisme est alors surjectif. 
	\end{remark}
	\begin{proof}
		$\varphi(1) = g$ donc $\varphi(n) = g^n , \forall n \in \Set{Z}$ soit $h \in G$, par d\'efinition on a $h = g^n$ pour un $n \in \Set{Z}$. \newline
		Donc $\varphi(n) = h$. 
		On a donc $\varphi$ surjectif
	\end{proof}
	Le morphisme $\varphi$ est-il injectif ? $\ker \varphi$ permet de r\'epondre. $\ker \varphi \sgr \Set{Z}$ donc $\ker \varphi = n\Set{Z}$.
	\begin{proposition}
		Soit $G$ un groupe monog\`ene, alors soit:
		\begin{enumerate}[(i)]
			\item	$G \simeq \Set{Z}$
			\item	$G \simeq \Set{Z} / n\Set{Z} , n > 0$
		\end{enumerate}
	\end{proposition}
	\begin{definition}
		Dans le second cas, le groupe est dit cyclique
	\end{definition}
	\begin{proof}
		On a vu que $\varphi : \Set{Z} \rightarrow G$ est surjective \newline
		Si $\ker \varphi = 0$ alors $\varphi$ est injective ,$\varphi$ est un isomorphisme et on est dans le cas \textit{(i)}. \newline
		Si $\ker \varphi \ne 0$ alors $\ker \varphi = n\Set{Z}$ \newline
		\fbox{Passage au quotient} \newline
		\fbox{Principe 1} Si $\varphi : G \rightarrow H$ est un morphisme, alors il d\'efinit un "isomorphisme canonique"
		\[ G/ \ker \varphi \simeq \im \varphi \]
		On obtient le diagramme commutatif suivant:
		
	\end{proof}	

